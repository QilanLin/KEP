% !TEX program = xelatex
\documentclass{beamer}
\usetheme{Madrid}
\usepackage{hyperref}
\usepackage{tikz}
\usetikzlibrary{arrows.meta,positioning,calendar}
\usepackage{fontspec}
% 注意:如果您看到中文字体显示为空白,请安装xeCJK包:
% sudo tlmgr install xecjk
% 然后使用下面的xeCJK版本(取消注释)

% 当前版本:使用fontspec设置中文字体(可能不完美)
\setsansfont{Heiti SC}
\setmainfont{Heiti SC}
\setmonofont{Menlo}

% 推荐版本:使用xeCJK包(需要先安装:sudo tlmgr install xecjk)
% \usepackage{xeCJK}
% \setCJKmainfont{Songti SC}
% \setCJKsansfont{Heiti SC}
% \setCJKmonofont{Heiti SC}

% 确保CJK字符正确换行
\XeTeXlinebreaklocale "zh"
\XeTeXlinebreakskip = 0pt plus 1pt

\title{相信我,我是验证器!\\\vspace{0.3em}Isabelle与外部证明器之间的接口可靠性}
\subtitle{Knowledge Exchange Projects - Amazon 项目 - Variant 3}
\author{Qilan Lin (K21204786)}
\institute{伦敦国王学院}
\date{2025--26}

\begin{document}

%------------------------------------------------
\begin{frame}
  \titlepage
\end{frame}

%------------------------------------------------
\begin{frame}{议程}
\begin{enumerate}
  \item 问题与动机
  \item 背景:Sledgehammer 流程
  \item 项目目标与研究问题
  \item 提议方法:接口模糊测试
  \item 预言器、评估、风险、时间线
\end{enumerate}
\end{frame}

%------------------------------------------------
\begin{frame}{问题陈述与动机}
\begin{block}{为什么这很重要}
Sledgehammer 是 Isabelle/HOL 与外部 ATP/SMT 证明器之间的主要桥梁。它增强了证明自动化,但依赖于复杂的接口层。
\end{block}

\begin{itemize}
  \item 接口缺陷可能导致崩溃、挂起,或"已找到证明"但无法重构的结果。
  \item 翻译/编码选择(例如,\texttt{lam\_trans}、\texttt{type\_enc})在功能与鲁棒性之间进行权衡。
  \item 现有测试专注于求解器;Isabelle\(\leftrightarrow\)证明器接口测试不足。
\end{itemize}

\vspace{0.3em}
\textbf{目标:}系统地压力测试此接口以提高可靠性。
\end{frame}

%------------------------------------------------
\begin{frame}{背景:Sledgehammer 概览}
\begin{center}
\scalebox{0.64}{
\begin{tikzpicture}[
  node distance=0.9cm,
  every node/.style={font=\small, align=center},
  box/.style={draw, rounded corners, minimum width=2.2cm, minimum height=0.9cm}
]

\node[box] (goal) {HOL 目标\\+ 上下文};
\node[box, right=of goal] (facts) {相关性\\事实选择};
\node[box, right=of facts] (encode) {HOL \(\to\) ATP/SMT\\编码};
\node[box, right=of encode] (prover) {外部\\证明器};
\node[box, right=of prover] (parse) {输出\\解析};
\node[box, right=of parse] (recon) {证明\\重构};

\draw[-{Latex}] (goal) -- (facts);
\draw[-{Latex}] (facts) -- (encode);
\draw[-{Latex}] (encode) -- (prover);
\draw[-{Latex}] (prover) -- (parse);
\draw[-{Latex}] (parse) -- (recon);

\end{tikzpicture}
}
\end{center}

\begin{itemize}
  \item Variant 3 针对步骤 \textbf{3--5}:编码、证明器 I/O 和重构。
  \item 这些步骤是通常出现不匹配和鲁棒性缺陷的地方。
\end{itemize}
\end{frame}

%------------------------------------------------
\begin{frame}{接口缺陷所在位置}
\begin{columns}[T]
\column{0.48\textwidth}
\begin{block}{SUT-A:编码/翻译}
\begin{itemize}
  \item HOL \(\to\) TPTP / SMT-LIB
  \item 类型与 \(\lambda\) 抽象编码
  \item 风险:格式错误或不够可靠的编码
\end{itemize}
\end{block}

\column{0.48\textwidth}
\begin{block}{SUT-B/C:证明器 I/O 与重构}
\begin{itemize}
  \item 进程调用、超时、解析
  \item Metis/SMT 证明重放
  \item 风险:"已证明"但不可重放、解析器崩溃
\end{itemize}
\end{block}
\end{columns}

\vspace{0.4em}
\textbf{工作假设:}这些层包含可通过系统变异输入利用的潜在边界情况缺陷。
\end{frame}

%------------------------------------------------
\begin{frame}{项目目标}
\begin{block}{高层目标}
构建一个针对 Isabelle 与外部证明器接口的模糊测试器,并定量提高其缺陷发现/测试覆盖率。
\end{block}

\textbf{交付物}
\begin{itemize}
  \item 一个专注于接口的模糊测试框架和种子语料库。
  \item 一组最小化的缺陷揭示实例 + 复现脚本。
  \item 相对于基线 Sledgehammer 使用的评估。
\end{itemize}
\end{frame}

%------------------------------------------------
\begin{frame}{研究问题}
\begin{itemize}
  \item \textbf{RQ1:}Sledgehammer 问题的结构感知变异能否发现接口中的崩溃/挂起?
  \item \textbf{RQ2:}变异问题触发证明器不一致(差异缺陷)的频率如何?
  \item \textbf{RQ3:}哪些编码/设置与重构失败相关?
  \item \textbf{RQ4:}模糊测试器在缺陷产出和接口路径多样性方面是否优于基线随机/未改变测试套件?
\end{itemize}
\end{frame}

%------------------------------------------------
\begin{frame}{方法概述}
\begin{enumerate}
  \item \textbf{种子提取:}直接在 AFP 目标上使用 Sledgehammer 的 CLI 导出 ATP/SMT 问题(通过 \texttt{sledgehammer\_export})。
  \item \textbf{结构感知变异:}生成语法上有效的变体(字典 + 语法约束)。
  \item \textbf{执行:}通过相同的 Sledgehammer 接口并行运行多个证明器。
  \item \textbf{预言器:}崩溃/挂起、差异不一致、重构失败。
  \item \textbf{最小化:}使用增量调试缩小缺陷实例。
\end{enumerate}
\end{frame}

%------------------------------------------------
\begin{frame}{模糊测试器定位(已确定)}
\begin{block}{主线(黑盒/灰盒)}
\textbf{在导出的 TPTP / SMT-LIB 问题上进行文件边界模糊测试。}
\end{block}

\begin{itemize}
  \item 输入正是接口所消费的内容(导出的问题 + 证明器输出)。
  \item 无需深入的 PolyML/Scala 插桩即可实现快速迭代。
  \item \textbf{MVP 范围:}从 \textbf{SMT-LIB} 导出 + SMT 证明器(Z3、cvc5)开始;
        \textbf{扩展:}稳定后添加 TPTP 导出 + ATP(E、Vampire)。
  \item 可选扩展:ML 编码器的进程内模糊测试。
\end{itemize}
\end{frame}

%------------------------------------------------
\begin{frame}{种子生成}
\begin{itemize}
  \item 选择代表性的 AFP 会话和目标。
  \item 直接使用 Sledgehammer 的 CLI 进行脚本化批量导出:
  \begin{itemize}
    \item 多个证明器(E、Vampire、Z3、cvc5、\dots)
    \item 编码开关:\texttt{lam\_trans}、\texttt{type\_enc}、时间片
    \item 利用 \texttt{sledgehammer\_export\_smt} 或自定义 ML 脚本
  \end{itemize}
  \item \textbf{阶段 1 种子:}优先处理 SMT-LIB 语料库以引导 AST 感知变异。
  \item 存储种子 + 元数据(目标 id、编码、预期状态)。
\end{itemize}

\vspace{0.3em}
\textbf{结果:}一个多样化的真实世界接口输入语料库。
\end{frame}

%------------------------------------------------
\begin{frame}{变异策略}
\begin{block}{结构感知变异模糊测试}
受黑盒变异 SMT 模糊测试器(例如 STORM)和类型/语法感知变异启发。
\end{block}

\begin{itemize}
  \item \textbf{AST 优先流程:}将 SMT-LIB/TPTP 种子解析为 AST,在树上变异,然后重新序列化
        以保持 \(\ge\)90\% 的变异体在语法上有效。
  \item 标记级编辑:重命名符号、翻转量词、扰动数字。
  \item 树级编辑:交换子项、复制/删除子句。
  \item 约束层保持公式类型良好且有效:
  \begin{itemize}
    \item 括号与绑定器平衡
    \item 每个 SMT 理论 / TPTP 片段的语法/字典
    \item 类型感知运算符替换(避免琐碎的类型错误垃圾)
  \end{itemize}
\end{itemize}
\end{frame}

%------------------------------------------------
\begin{frame}{相关工作(定位)}
\begin{itemize}
  \item \textbf{Sledgehammer 接口与重放失败:}
        Sledgehammer 经常报告"已找到证明"但重构可能失败
        (例如,"单行证明重构失败"),这促使进行接口级测试。
  \item \textbf{黑盒 SMT 模糊测试:}
        STORM 显示结构感知变异 + 最小化可以暴露真实的 SAT/UNSAT 正确性缺陷。
  \item \textbf{类型/语法感知差异测试:}
        类型感知 AST 变异和跨求解器检查在发现正确性缺陷方面非常有效。
\end{itemize}
\end{frame}

%------------------------------------------------
\begin{frame}{缺陷预言器}
\begin{itemize}
  \item \textbf{崩溃/挂起预言器}
  \begin{itemize}
    \item Isabelle/Sledgehammer 异常、证明器崩溃或硬超时。
  \end{itemize}
  \item \textbf{差异预言器}
  \begin{itemize}
    \item 相同输入,不同证明器/设置返回不一致结果。
    \item \textbf{噪声过滤器(SMT 差异测试标准):}
      \begin{itemize}
        \item 仅将 \textbf{SAT vs UNSAT / 已证明 vs 已反驳} 计为正确性冲突;
        \item 将 \texttt{unknown}/超时视为性能信号,而非缺陷;
        \item 重新运行冲突以确认可重现性。
      \end{itemize}
  \end{itemize}
  \item \textbf{重构预言器}
  \begin{itemize}
    \item 外部证明器成功但 Isabelle 重放失败(官方已知的 Sledgehammer 失败类别)。
    \item 按编码旋钮和 Metis/SMT 重放策略跟踪失败率。
  \end{itemize}
\end{itemize}

\vspace{0.3em}
每个预言器产生一个保存的测试用例用于最小化。
\end{frame}

%------------------------------------------------
\begin{frame}{最小化与报告}
\begin{itemize}
  \item 使用增量调试缩小失败实例。
  \item 归一化为最小的可重现核心(供开发者分类)。
  \item 生成缺陷报告包:
  \begin{itemize}
    \item 种子 id + 变异轨迹
    \item 证明器日志和 Isabelle 重构轨迹
    \item 单命令复现器
  \end{itemize}
\end{itemize}
\end{frame}

%------------------------------------------------
\begin{frame}{实现计划}
\begin{itemize}
  \item \textbf{驱动:}基于 Python 的模糊测试器运行器。
  \item \textbf{Isabelle 钩子:}导出问题,一致地调用证明器。
  \item \textbf{证明器池:}本地安装 + Isabelle 包装器。
  \item \textbf{存储:}语料库、覆盖率代理、失败数据库。
\end{itemize}

\vspace{0.3em}
\textbf{原型里程碑:}在小 AFP 子集上工作的端到端循环。
\end{frame}

%------------------------------------------------
\begin{frame}{MVP 演示计划(30--60 秒)}
\begin{enumerate}
  \item \textbf{从 AFP 目标导出真实的 SMT-LIB 种子}通过 Sledgehammer CLI
  \begin{itemize}
    \item 显示 1 个脚本/命令 + 生成的 \texttt{.smt2} 文件(元数据:目标、编码旋钮)。
  \end{itemize}

  \item \textbf{运行模糊测试器小预算}(例如,100--500 个变异体,Z3 vs cvc5)
  \begin{itemize}
    \item 屏幕显示:生成变异体 $\rightarrow$ 调用证明器 $\rightarrow$ 收集日志。
  \end{itemize}

  \item \textbf{显示 1 个具体发现}以及预言器如何触发
  \begin{itemize}
    \item 要么 \textbf{(a)} SAT/UNSAT 冲突通过重新运行确认,或
    \item \textbf{(b)} "已证明"但 \textbf{重放失败}(官方 Sledgehammer 失败类别)。
    \item {\scriptsize 差异经验法则:仅将 \textbf{SAT vs UNSAT} 计为正确性缺陷;
          \texttt{unknown}/超时被过滤为性能信号。}
  \end{itemize}

  \item \textbf{最小化输出}
  \begin{itemize}
    \item 显示之前/之后大小 + 缺陷包中的单命令复现器。
  \end{itemize}
\end{enumerate}
\end{frame}

%------------------------------------------------
\begin{frame}{评估}
\begin{block}{基线}
在相同目标上运行 Sledgehammer 而不进行变异(标准批量测试)。
\end{block}

\textbf{指标}
\begin{itemize}
  \item 发现的唯一崩溃/挂起。
  \item 发现的唯一差异正确性冲突。
  \item 按编码 / 重放策略的重构失败率。
  \item \textbf{接口路径覆盖率代理:}
    \begin{itemize}
      \item 在编码、证明器 I/O、解析、重放中触发的唯一异常位置 / 结果类别数量;
      \item 比较基线 vs 模糊测试器以显示新探索的接口路径。
    \end{itemize}
  \item 首次缺陷时间、每小时缺陷数、语料库多样性。
\end{itemize}
\end{frame}

%------------------------------------------------
\begin{frame}{风险与缓解}
\begin{itemize}
  \item \textbf{过多无效变异体} $\rightarrow$ AST/语法约束 + 修复。
  \item \textbf{搜索空间爆炸} $\rightarrow$ 按新颖性和失败历史优先排序。
  \item \textbf{工具/安装摩擦} $\rightarrow$ 固定 Isabelle 版本 + Docker 后备方案。
  \item \textbf{难以分类的失败} $\rightarrow$ 积极最小化和完整日志。
\end{itemize}
\end{frame}

%------------------------------------------------
\begin{frame}{预期贡献}
\begin{itemize}
  \item 第一个专注于 Isabelle\textendash 证明器接口的系统化模糊测试器。
  \item 一个可重用的 Sledgehammer 导出问题种子语料库。
  \item 缺陷报告 / 补丁,提高稳定性和可信度。
  \item \textbf{哪些编码旋钮与重放失败相关}的经验映射。
\end{itemize}
\end{frame}

%------------------------------------------------
\begin{frame}{时间线(指示性)}
\begin{center}
\scalebox{0.75}{
\begin{tikzpicture}[
    gantt/.style={draw, fill=blue!30, minimum height=0.6cm, rounded corners=2pt},
    ganttlabel/.style={anchor=east, font=\scriptsize},
    weeklabel/.style={font=\tiny, anchor=north}
]
    % Week labels
    \foreach \x in {0,...,12} {
        \node[weeklabel] at (\x*0.8, 4) {\x+1};
    }
    \node[anchor=south east, font=\small] at (-0.5, 4.5) {周};
    
    % Task bars
    \node[ganttlabel] at (-0.5, 3.5) {\scriptsize 文献综述、环境搭建};
    \draw[gantt] (0, 3.3) rectangle (1.5, 3.7);
    
    \node[ganttlabel] at (-0.5, 3.0) {\scriptsize Sledgehammer 流程研究};
    \draw[gantt] (2, 2.8) rectangle (3.5, 3.2);
    
    \node[ganttlabel] at (-0.5, 2.5) {\scriptsize AST 变异引擎 + 预言器};
    \draw[gantt] (4, 2.3) rectangle (6.5, 2.7);
    
    \node[ganttlabel] at (-0.5, 2.0) {\scriptsize 大规模模糊测试活动};
    \draw[gantt] (7, 1.8) rectangle (8.5, 2.2);
    
    \node[ganttlabel] at (-0.5, 1.5) {\scriptsize 最小化、缺陷分类、分析};
    \draw[gantt] (9, 1.3) rectangle (10.5, 1.7);
    
    \node[ganttlabel] at (-0.5, 1.0) {\scriptsize 撰写、幻灯片、最终评估};
    \draw[gantt] (11, 0.8) rectangle (12.5, 1.2);
    
    % Grid lines
    \foreach \x in {0,...,13} {
        \draw[gray!30, dashed] (\x*0.8, 0.5) -- (\x*0.8, 4.2);
    }
\end{tikzpicture}
}
\end{center}
\end{frame}

%------------------------------------------------
\begin{frame}{时间线(详细)}
\begin{tabular}{ll}
第 1--2 周 & 文献综述、环境搭建 \\
第 3--4 周 & Sledgehammer 流程研究、种子导出脚本 \\
第 5--7 周 & AST 变异引擎 + 预言器、端到端原型 \\
第 8--9 周 & 大规模模糊测试活动 \\
第 10--11 周 & 最小化、缺陷分类、分析 \\
第 12--13 周 & 撰写、幻灯片、最终评估 \\
\end{tabular}
\end{frame}

%------------------------------------------------
\begin{frame}{总结}
\begin{itemize}
  \item Variant 3 针对一个高影响但测试不足的层。
  \item 主要方法:文件边界、AST/语法感知变异模糊测试。
  \item 强预言器 + 系统评估使结果可发表。
\end{itemize}

\vspace{0.5em}
\textbf{立即下一步}
\begin{itemize}
  \item 实现 Sledgehammer CLI 脚本以进行批量种子提取
  \item 实现核心变异操作符(AST 感知)
  \item 设置证明器池和测试框架
  \item 在 SMT-LIB 语料库上开始阶段 1 评估
\end{itemize}
\end{frame}

\end{document}

